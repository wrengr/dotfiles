%  Exercise
%
%  Copyright (C) 2001, 2002, 2003, 2007 Didier R�my
%
%  Author         : Didier Remy 
%  Version        : 1.1.2
%  Bug Reports    : to author
%  Web Site       : http://pauillac.inria.fr/~remy/latex/
% 
%  Exercise is free software; you can redistribute it and/or modify
%  it under the terms of the GNU General Public License as published by
%  the Free Software Foundation; either version 2, or (at your option)
%  any later version.
%  
%  Exercise is distributed in the hope that it will be useful,
%  but WITHOUT ANY WARRANTY; without even the implied warranty of
%  MERCHANTABILITY or FITNESS FOR A PARTICULAR PURPOSE.  See the
%  GNU General Public License for more details 
%  (http://pauillac.inria.fr/~remy/license/GPL).
%
%%%%%%%%%%%%%%%%%%%%%%%%%%%%%%%%%%%%%%%%%%%%%%%%%%%%%%%%%%%%%%%%%%%%%%%%%%%%
%  File exercise.tex (LaTeX macros)
%%%%%%%%%%%%%%%%%%%%%%%%%%%%%%%%%%%%%%%%%%%%%%%%%%%%%%%%%%%%%%%%%%%%%%%%%%%%

\def \Version {1.1.2}
\def \Modified {2007/08/28}

\documentclass[12pt] {article}
\usepackage {fullpage}

%\usepackage[USenglish,francais]{babel} 

\usepackage {hevea}
\def \cuttingunit {none}

\usepackage{exercise} 



%% The following line is possible in latex but would not be understood by hva
% \usepackage [foo]{exercice}

%% You may uncomment the following lines see my hyperref.hva package
\usepackage {hyperref}


\title {Sending \\
Answers to Exercises \\
(or Proofs of Theorems) \\ 
to appendices \\[1em]
\normalsize (Version {\Version}, last modified {\Modified})}
\author {Didier R{\'{e}}my}


\begin{document}

\maketitle

\begin{abstract}
This package provides macros for writing exercises and answers to
exercises. Then, the same document can be run in several modes, for
instance, the answers may be hidden, inserted on place, or just collected 
in an auxilliary file to be inserted later in an appendix for instance. 
(Then  hyperref can be used to make questions and answers point to one
another.) 

The package can also be used to send proofs of theorems to appendices. 

The package also comes with a version for {\hevea}.
\end{abstract}

\ifhevea
\def \basis {..}% {http://cristal.inria.fr/~remy/latex}
This document is also available in
\href {\basis/exercise.dvi}{DVI}, 
\href {\basis/exercise.ps.gz}{Postscript}, and
\href {\basis/exercise.pdf}{PDF}.

The source of this file,
\ahref {\basis/exercise.tex}{exercise.tex}
may serve as an example.
\fi


%% This is an example of use of anser outside of exercises

\begin{answer}
\em Answers to exercises are intentionlly very brief.
\end{answer}

%% This is an interesting use of \fragileanswer (see the documentation)
%% \let \oldsection \section
%% \renewcommand {\section}[1]
%%   {\oldsection {#1}\fragileanswer
%%      {\string \subsection* {Section \thesection\space (#1)}}}

%\AnswerLaterfalse
%\Answersfalse

\section {The exercise and answer environment}

The \verb"exercise" environment is used to separate and number exercises. 
The environment contains an optional parameter meant to be a short
description of the exercise that will then appear as the title of the
exercise, when refered to from another position.

The \verb"answer" environment is normally used within the \verb"exercice"
environment, in which case the exercise number is captured and used to refer
to the exercise from another position.  There may be several answers within
a single exercise environment.  Answers are numbered internally ({\em ie.}
independently of any question and subsection structure that the author could
introduce) and its number refers to the $n$-th answer envrionment from the
beginning of the exercise.


\subsection {Collecting answers into an answer-file (default mode)}

The default behavior of the \verb"answer" environment is to copy
the contend of the answer (the region of text between \verb"\begin{answer}"
and \verb"\end{answer}" verbatim to the answer file
\verb"\ExerciseAnswerFile.ans"
(by default \verb"\ExerciseAnswerFile" is equal to \verb"\jobname")

The \verb"answer" environment can also be used outside of the
\verb"exercise" environment, in which case the text is inserted verbatim in
the answer file, but without any title and any other side effect.  For
instance, this may be used to insert in the answer files some headers or
notes that are global to several exercises.

The fragile command \verb"\fragileanswer" is an alternative to the use 
of the use of the  \verb"answer" environment. However, since it evaluates
its arguments, commands and such must be quoted. The advantage is that it
can be used inside other macros.

The command \verb"\inputanswers" normally insert all answers to
exercises. It takes the title of the answer section as a mandatory
argument. The base file name of the answer file, may be given as an optional
argument; this allows to load answers generated from another file.


\subsection {Customizing answer placements}

Customization of answer placement uses the \verb"keyval" package.
The command \verb"\setanswer" can be used with the following keys:

\medskip\noindent%\hspace {-2em} 
\verb"noanchors"
\begin{quote}
This will remove hyperref anchors to exercises (which are set by default).
\end{quote}

\medskip\noindent%\hspace {-2em} 
\verb"anchors"
\begin{quote}
This will attached hyperref anchors to exercises (if previously removed). 
\end{quote}

\medskip\noindent%\hspace {-2em} 
\verb"hide"
\begin{quote}
Answers will be hidden. 
\end{quote}

\medskip\noindent%\hspace {-2em} 
\verb"show"
\begin{quote}
Answers will be shown. 
\end{quote}

\medskip\noindent%\hspace {-2em} 
\verb"later"
\begin{quote}
If shown, answers will appear later. 
\end{quote}

\medskip\noindent%\hspace {-2em} 
\verb"inline"
\begin{quote}
If shown, answers will appear immediately. 
\end{quote}

\medskip\noindent%\hspace {-2em} 
\verb"file="{\tt \em filename}
\begin{quote}
This will use filename as the base name for the answer file. 
By default \verb"\jobname" is used. The suffix \verb".ans" will be appended
to the basename. This command key can only be set in the preamble. 
\end{quote}

The command \verb"\inputanswers" can be used to insert answers that have been
postponed. The command takes one argument which is used as the title of the
answer section before answers are inserted. The command also takes an
optional argument which is the name of the answer file. 
By default the filename if the same one as the one in which answers were
collected. If answers are ignore (as set by the \verb"hide"  key) or
immediate (as set by the \verb"inline" key) at the time
\verb"\inputanswers" is executed, then the command is ignored (hence, the
answer section is not created). 

These keys can be set globally with the command \verb"\setanswers". 
The can also be modified globally anywhere or locally in the scope of
grouping parens. Furthermore, they can be set locally to an exercise 
by appearing in an optional argument to the exercise environment. 

\subsection {Package options}

The keys \verb"noanchors", \verb"hide", and
\verb"now" are also recognized as package options. 
Any other package option is interpreted as the argument to the key
\verb"file". 

\subsection {Example}

\begin{exercise}
What would be the effect of the following declaration?
\begin{verbatim}
\let \oldsection \section
\renewcommand {\section}[1]
  {\oldsection {#1}\fragileanswer
     {\string \subsection* {Section \thesection\space (#1)}}}
\end{verbatim}
\begin{answer}
See the section information inserted above and below in the answer section.
\end{answer}
(Of course the code need improvement to take the optional arguments sections
into account)
\end{exercise}
 

\begin{exercise}
Note that this exercise has no title.
\par \noindent
1. what will you see at this place in the main body of the
document?
\begin{answer}
The question above, but you will not see this.
\end{answer}
2. Where will I see the answer to the previous question and to this
one. 
\begin{answer}
At the place where you insert \verb"\inputanswers"
\end{answer}
\end{exercise}
 
\begin{exercise}[Hiding answers]
What would happen if I type \verb"\setanswer{noanchor}" in the body of this
exercise?  
%\setanswer{noanchor}
\begin{answer}
The answer would be shown as usual in the list of answers, but there won't be
an anchor from the exercise to its solution. 
Note that this would also affect following answers of the same exercise. 
\end{answer}
And what will happen if I now type \verb"\setanswer{hide}" in the body of
this exercise?
%\setanswer{hide}
\begin{answer}
This answer would be invisible, as if removed from the file. 
\end{answer}
\end{exercise}

\begin{exercise}[Input Answers]
What will happen if the command \verb"\inputanswers" is never included?
\begin{answer}
The answer of exercises will just not be shown and (anchors to answers
will be dandling pointers)
\end{answer}
Note that the file \verb"\jobname.ans" is only created if non empty.
Thus, if you load the package but do not create any exercise it is not
created. Then, you will simply get a warning if you attempt to load answers
of exercises. 
\end{exercise}

\subsection {Customizing the appearance of answers}

This can be done by redefining the following commands:

\medskip\noindent%\hspace {-2em} 
\verb"\AnswerSection"
\begin{quote}
The name of the command to print the title of answer section (the argument
of \verb"\inputanswers". 
\end{quote}

\medskip\noindent%\hspace {-2em} 
\verb"\ExerciseName"
\begin{quote}
The identifier used to name exercises (often need to be changed according to
languages). Default value is {\tt Exercise}, except if
\verb"\languagename" is {\tt frenchb} where the value is
{\tt Exercice}. 
\end{quote}

\medskip\noindent%\hspace {-2em} 
\verb"\AnswerName"
\begin{quote}
The identifier used to point to answers. Default value is {\tt Answer},
except if \verb"\languagename" is {\tt frenchb} where the value is
{\tt R{\'e}ponse}. 
\end{quote}

\noindent%\hspace {-2em} 
\verb"\AnswerNew"
\begin{quote}
The title of an answer to a new exercise. It receives 3 arguments: the exercise
number, the answer number, and the page number. 
\end{quote}

\noindent%\hspace {-2em} 
\verb"\AnswerContinued"
\begin{quote}
Idem, but when the answer is not the first one.
\end{quote}

\noindent%\hspace {-2em} 
\verb"AnswerNow" 
\begin{quote}
This environment can be redefined to change the typesetting
immediate answers (when answers are immediate, Answer will act as AnswerNow).
\end{quote}

\section {Creating new answer environments}

In addition to customization of the answer environment, the user can defined
its own answer environment. For instance, an environment \verb"myanswer"
similar to the default \verb"answer" of this package can be defined with the
following definition: 
\begin{verbatim}
\newanswer {myanswer}{Answer}
  {\em \AnswerNiceLink}
  {}
  {\AnswerNow}
  {\endAnswerNow}
  {{\thepage}}
  [1]
  {\ifnum \theQuestion=1\relax 
     \subsection* 
        {\ExerciseLink {Exercise \theExercise, page #1}}%
   \else
     \subsubsection* 
        {\ExerciseLink {Exercice \theExercise\space (continued)}}%
   \fi}
  {}
\end{verbatim}
The command \verb"\newanwer" takes at least 8 arguments, with two possible
optional arguments after the fourth one. 
\begin{itemize}
%HEVEA\let \Item \item\def \item[#1.]{\Item{}{\bf \##1:}}

\item[1.]
The first argument is the name of the answer environment being defined. 

\item[2.]
The second argument is the title of the answer environment to appear in 
the document.

\item[3, 4.]
This are will be executed respectively before and after processing the
answer.  Typically, they are used to insert a pointer to the exercise in the
appendix.

\item[5, 6.]
These define the environment that should be used in case the answer is
inserted locally, {\em ie.} if \verb"\AnswerLaterfalse" is set.
Two optional arguments may be used between mandatory arguments 4 and 5
and will act as in the {\tt\string\newenvironment} command. 

\item[7.]
This is  used to pass information between the place
where the answer environment occurs and the time when the answer will be
reread again. This argument is passed to the \verb"\fragileanswer", {\em ie.}
it is evaluated in the local context but inserted in the answer file. 

\item[8, 9.]
The remaining arguments are used to define the behavior of the answer 
when reread from the answer file. They are passed to the command
{\tt \string \newenvironment}\verb"{ans-name}" (where \verb"name" is the name
of the answer-environment being defined) hence the two optional arguments in
case this command  used arguments (possibly with default values). 

\end{itemize}
Additionally, the so-created \verb"myanswer" environment that is 
immediately passed to \verb"\setanswer" for local customization. 
This is evaluated first, so that it act as if it \verb"\setanswer"
were executed outside of the exercise environement, except that the setting
if local to the environment. 

% %%%%%%%%%%%%%%%% For use in this file.
% \newanswer {myanswer}{Answer}
%    {\em \AnswerNiceLink}
%    {}
%    {\AnswerNow}
%    {\endAnswerNow}
%    {{\thepage}}
%    [1]
%    {\ifnum \theQuestion=1\relax 
%       \subsection* {\ExerciseLink {Exercise \theExercise, page #1}}%
%     \else
%       \subsubsection* 
%          {\ExerciseLink {Exercice \theExercise\space (continued)}}%
%     \fi}
%    {}
% %%%%%%%%%%%%%%%% 
More precisely, at every of the follow code in the source file:
\begin{quote}
\begin{verbatim}
\begin{myanswer}[<args>]
<Any text>
\end{myanswer}
\end{verbatim}
\end{quote}
The following steps are executed in order ($\texttt{\#}k$ refers to the k's
argument of the \texttt{myanswer} environment definition): 
\begin {enumerate}
\item{}
As at the beginning of any environment, a new group (scope) is opened

\item{}
The optional argument \verb"<args>" of the environment is passed to
\verb"\setanswer". Unless the answer is hidden or immediate, the following
steps continues as folllow (otherwise, the environment is ignore or inlined 
according to arguments {\tt \#4} and {\tt \#5}).

\item{}
\verb"\begin{ans-myanswer}" is inserted in the answer file.

\item{}
The argument \texttt{\#6} is inserted into the answer file. 

\item
The argument \texttt{\#2} is evaluated. 

\item{}
The body of the answer \verb"<Any text>" is inserted verbatim into the answer
file. 

\item{}
The argument \texttt{\#3} is evaluated.

\item{}
\verb"\end{ans-myanswer}" is inserted in the answer file.

\item{}
The group (scope) of the environment is closed.

\end {enumerate}
In addition, two commands are available for manipulating anchors:
\verb"\AnswerLink" can be used in the text to point to the corresponding 
answer in the answer section and, conversely, \verb"\ExerciseLink" can be
used to point to the exercise from the answer section. Both commands take 
one argument (the text where to anchor the link). 
A more elaborated version \verb"\AnswerNiceLink" provides
a default typesetting of the answer link. It takes an optional parameter
---the text of the anchor--- equal to \verb"\AnswerName" by default. 

The environment also adjusts the counters \verb"\Exercise" and
\verb"\Question" so that their values while reading the answer file is
identical to the value they had when the corresponding answer was processed.
Other counters (such as pages, sections, etc.) must be passed explicitly 
using the fourth argument, as described above. 

These steps only applies is the answer appears within an exercise, 
and if answers are not hidden. Otherwise, the environment behaves as 
in the default case. 

% \paragraph*{Warning!} Immediate answers are inserted, 
% as in the default environment; hence, any change of style made by the answer
% envrironement is ignored in this situation and the typesetting may not
% correspond to the one when the answer appears in an appendix.

\section {Sending proofs to an appendix}

This package can also be used to typeset proofs locally right after 
the claims, and have them appear in an appendix.

For instance this can be realized with the following declaration:
\begin{verbatim}
\def \capitalize #1{\uppercase {#1}}
\newanswer {proof}{Proof}
  {\def \theExercise {\ThmName}\def \theQuestion {\ThmLabel}%
   \if@answer@later
       \if@answer@anchor \AnswerLink{\hbox {(See proof in the Appendix)}}\fi%
   \else \def \AnswerName {Proof}\fi
   \pdf@hypertarget {Thm-\ThmName.\ThmLabel}{}}
  {}
  {\trivlist \item [\hskip \labelsep {\underline {Proof}:}]}
  {\penalty 100\hfill\rule{2mm}{2mm}\endtrivlist\@doendpe}
  {{\ThmName}{\ThmLabel}}
  [2]
  {\subsection* 
        {\pdf@hypertarget {Ans-#1.#2}{Proof} of 
         \pdf@hyperlink {Thm-#1.#2}{\capitalize #1 #2}}%
   \nobreak
   }
  {}
\end{verbatim}
Note the use of \verb"\ThmName" and \verb"\ThmLabel", which refer to the
name and counters of the last theorem-like environment.  They can be
retrieved automatically will the following little hack.
\begin{verbatim}
\def \@thm #1#2{%
  \refstepcounter {#1}
  \xdef \ThmName {\@currenvir}%
  \xdef \ThmLabel {\@currentlabel}%
  \@ifnextchar [{\@ythm {#1}{#2}}{\@xthm{#1}{#2}}%
}
\end{verbatim}
Then, proofs will be recalled by 
\begin{verbatim}
\inputanswers {Proof of main results}
\end{verbatim}
\paragraph {Warning!}
Although the package can be used to pospone several kinds of material to an
appendix using a temporary file, it uses a single file so far. So combining
solutions to exercises and proofs to appendices will not work together 
(material would be interlaced). Sometimes interlacing may be desired, but
usually not. In the future, the command should offer the possibility to
send different kind of material to different files. 

When,  \verb"\inputanswers" is called, all pending answers are inserted. 
If other (exercises and) answers occur after \verb"\inputanswers", 
the file will be reopened as an empty file and new answers will be 


\section {Implementation notes}

The implementation is quite simple and short. 
%
The only trick is to undo the active catcodes that might have been done by the
\verb"inputencoding" package before copying answers to 
\verb"\ExerciseAnswerFile.ans".

\subsubsection* {Beware!}

However, I found out, that to make it work you also need to tell latex to
recognize 8-bit characters, i.e. to treat ``\'{e}'' as such, and not as
``\verb"^^e9"''. On my redhat 6.1 linux machine, it is sufficient
to set the environment variable \verb$LC_ALL=en_US$.  I do not know any
otherway of making latex print 8-bit characters. 
Of course, printing ``\verb"^^e9"'' instead of ``\'e'' is quite problematic
with with verbatim code, where the ``\verb"^^"'' will be taken verbatim and
not as an encoding of an 8-bit character. The solution would them
be to pacth the verbatim mode to keep the meaning of ``\verb"^^"'' as 
encoding of 8-bit charaters. 

{\em Another, probably simpler, solution would then be to post-process the
answer file to recode 7-bit charaters into 8-bit characters. However, one
cannot reread the reshly created answer file. Instead two different names
for the output and input answer files should be used. }

\section {HeVeA compatibility}

The package also comes with some {\hevea} version. 
{\hevea} asumes that the file has first been processed by tex so that
\verb"\jobname.ans" already exists. 

However, {\hevea} does not currently interpret package options. 
So, you must set the corresponding flags by hand, instead, to ensure
{\hevea} compatibility.

 to change
the name of \verb"\jobname" does not currently work in {\hevea}, so you
will have to do the appropriate renaming by hand (or in a Makefile). 

The {\hevea} version is designed to be combined with hacha:
in addition to pointing to answers, the answers are also included 
immediately, but in a separate cutting unit, so that hacha will put them in
an auxilliary file. 



%\let \section \oldsection 

\inputanswers{Answers to exercises}

\section{We continue, and may include more exercises}

\begin{exercise}[This is an additional optional exercise]
Where the answer to this exercise will go?
\begin{answer}
In a new file \verb"\jobname1.ans". 
\end{answer}
How can we include answers of these additional exercises? 
\begin{answer}
By just repeating the \verb"\inputanswers" command.
\end{answer}
\end{exercise}


\inputanswers{Answers to additional exercises}

\end{document}
% LocalWords:  mathpar aa Yop bb dd ee ff cc Ocaml Foo leftskip rightskip vdots
% LocalWords:  gg hh uu vv ww HeVeA escapechar
