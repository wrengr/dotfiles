\documentclass{article}

\usepackage{hyperref}

%\usepackage{exercice}
\usepackage{proof}

\newtheorem{lemma}{Lemma}
\newtheorem{theorem}{Theorem}

\def \WhizzyTeX{Whizzy\TeX}

\begin{document}

\author{Didier R{\'e}my}
\title {A proof environment, as an exercise\\
together with its {\WhizzyTeX} companion}
\maketitle


This is a short, self-documnented article with lemmas and proofs.
Please read the source under {\WhizzyTeX} and the output 
at the same time.

Just move the cursor down the file in and out of proofs environment. 
You may also see the self-documented article \texttt{exercise.tex}
from the distribution.

\begin{verbatim}
\newtheorem{lemma}{Lemma}
\newtheorem{theorem}{Theorem}
\end{verbatim}

\begin{lemma}
This is the first lemma. 
\end{lemma}
Some short comment on the lemma.
\begin{proof}{}%inline
This is the proof of the previous lemma. 

That is the first one.

Its proof is not too long. 

It just first on a few lines. 

And it ends there. 
\end{proof}
The text continues after the proof.

\begin{lemma}
This is the second lemma. 
\end{lemma}
Some short comment on the lemma.
\begin{proof}{}%inline
This is the proof of the previous lemma. 
A  very long proof\ldots

\def \p{.\par}\def \pv{\p\p\p\p\p}\def \px{\pv\pv}
\px\px\px\px

That ends here. 
\end{proof}
The text continues after the proof.

\begin{theorem}
This is the second lemma. 
\end{theorem}
Its proof is a bit shorter. 
\begin{proof}{}%inline
This is the proof of the previous lemma. 
This is the proof of the previous lemma. 
\end{proof}
The text continues after the proof.

\begin{lemma}
Another lemma, without a proof.
\end{lemma}


\begin{lemma}
Yet another lemma, with its proof.
\end{lemma}
\begin{proof}{}
It the numbering right?
\end{proof}
\newpage
\inputanswers{Proofs}

\end{document}
