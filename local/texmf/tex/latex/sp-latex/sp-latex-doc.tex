%!TEX TS-program = pdflatexmk

\documentclass[lucida,final]{sp}

%\setcounter{errorcontextlines}{999} % useful for debugging

% The recommended examples package for S&P.
\usepackage{example}


%---------------------------------------------------------------------
% Used for this specific document; likely not of wider utility.
\usepackage{fancyvrb}
\fvset{xleftmargin=\parindent}
\newcommand{\BibTeX}{B{\textsc i\kern-.025em\textsc b}\kern-.08em\TeX}
\newcommand{\SandP}{S\&P\@}
\newcommand{\spdir}[1]{\texttt{#1}}
\newcommand{\spfile}[1]{\texttt{#1}}
\newcommand{\spcode}[1]{\texttt{#1}}
\newcommand{\cmd}[1]{\texttt{\textbackslash#1}}
\newcommand{\argcmd}[2]{\texttt{\textbackslash#1\{#2\}}}
\newcommand{\twoargcmd}[3]{\texttt{\textbackslash#1\{#2\}\{#3\}}}
%---------------------------------------------------------------------

%=====================================================================
%========================= preamble material =========================

% Metadata for the PDF output.
\pdfauthor{Kai von Fintel, Christopher Potts, and Chung-chieh Shan}
\pdftitle{Instructions for S&P authors using LaTeX2e}
\pdfkeywords{Semantics and Pragmatics, LaTeX2e, BibTeX, open acess journal}

% Optional short title inside square brackets, for the running headers.
% If no short title is given, no title appears in the headers.
\title[S\&P instructions for \LaTeXe]{Instructions for S\&P authors using \LaTeXe%
  \thanks{Our thanks to Donald Knuth, Leslie
    Lamport, and the other developers of \TeX\ and \LaTeX, for making
    it possible to produce documents like this without the aid of a
    publishing house.}}

% Optional short author inside square brackets, for the running headers.
% If no short author is given, no authors print in the headers.
\author[von Fintel, Potts \& Shan]{% As many authors as you like, each separated by \AND.
  \spauthor{Kai von Fintel \\ \institute{Massachusetts Institute of Technology}} \AND
  \spauthor{Christopher Potts \\ \institute{University of Massachusetts Amherst}} \AND
  \spauthor{Chung-chieh Shan \\ \institute{Rutgers University}}%
}

% Allow page breaks within a linguistic example.
\interdisplaylinepenalty=200

\spyear{2007}
\spvolume{0}
\sparticle{2}
\splastpage{14}

%=====================================================================

\begin{document}

%=====================================================================
%============================ frontmatter ============================

\maketitle

\begin{mshistory}
First published 2007-11-27. Latest version: 2011-12-20.
\end{mshistory}

\begin{abstract}  
  This document provides instructions for installing and using the
  \LaTeXe\ class file, \spfile{sp.cls}, and the \BibTeX\ formatting
  file \spfile{sp.bst}.
\end{abstract}

\begin{keywords}
  \emph{Semantics and Pragmatics}, \LaTeXe, \BibTeX, open-access journal
\end{keywords}

%=====================================================================
%============================ article text ===========================

\section{Introduction}

This document describes how to use the \LaTeXe\ files for \SandP.  The
next few sections explain how to install and work with the package,
which includes all the bells and whistles needed to produce high
quality web-ready PDF files.  Many of the macros are specific to
\SandP, but the package should be generally useful as well.

The source file for this document is a useful example, since it
employs most of the features of the class file and bst file.  You can
also download a template file from the \SandP\ website.  We hope,
though, that you're able to fairly seamlessly use your current source
files, but with \spfile{sp} as the documentclass and the \SandP\
specific frontmatter added.

Give \LaTeXe\ a try if you haven't already converted.  It's a shame to
work for months or years on a paper only to have it come out looking
like, well, like a Word document.  So make the switch if you haven't
already.  The classic reference for \TeX\ is \citealt{Knuth84TeXBook},
and the classic reference for \LaTeX\ is \citealt{Lamport94}.  But
skip those if you are just starting out.  We recommend instead the
following newer book, which covers the more recently \LaTeXe\ releases
and will serve you well even when you're an expert:
\citealt{KopkaDaly03}.  (Don't worry about the distinctions between
\TeX, \LaTeX, and \LaTeXe\ \dash you're bound to end up with \LaTeXe\ if
you install something now.)

%=====================================================================
%=====================================================================

\section{Installation}

%=====================================================================

\subsection{Package contents}\label{sec:contents}

The \SandP\ package consists of five files:
%
\example{
  \subexamples{
    \subexample{\spfile{sp.cls}}
    \subexample{\spfile{sp.bst}}
    \subexample{\spfile{example.sty}}
    \subexample{\spfile{mdwtab1.sty}}                         
    \subexample{\spfile{wokluwer.sty}}
  }
}
%
Only two of these files is essential to using the package:
%
\example{\label{ex:contents}Essential files\\
  \subexamples{
  \subexample{\label{ex:cls}\spfile{sp.cls}: the class file}
  \subexample{\spfile{sp.bst}: the \BibTeX\ style file}}}
%
You should put these where your \LaTeX\ installation looks for
such files (say, your local \spdir{/texmf/tex/latex} for the cls
file and \spdir{/texmf/bibtex/bst} for the bst file).

The file \spfile{example.sty} is for typesetting examples.  It
requires \spfile{mdwtab1.sty} and \spfile{wokluwer.sty}, also
included.  \spfile{example.sty} is not loaded by the class file, so it
should be called with \cmd{usepackage} if you want to use it.  Section
\ref{sec:examples} discusses examples in more detail.

%=====================================================================

\subsection{Requirements}

The class file loads a number of prerequisite packages.  All of them
are included in standard \LaTeXe\ distributions.  If you happen not to
have one of them, then it will be freely downloadable from
\href{http://ctan.org}{ctan.org}.

The official fonts of \SandP\ are the Lucida fonts designed by
\citet{bigelow-holmes:1986:integrated,bigelow-holmes:2005:lucida-designs}. However,
since these fonts are commercial rather than freely available, the
package will use Times unless you select the \spcode{lucida}
option. This means, unfortunately, that you will not be able to see
where line-breaks and page-breaks will occur in the final typeset
version unless you happen to own the Lucida fonts.

The other packages loaded by the class file are standard, so you
probably don't need to do anything special for them:
%
\examples[l@{}l@{}X]{%
\ex & \multicolumn{2}{@{}X@{}}{Packages loaded by \spfile{sp.cls}} \\
&\ey& \spfile{fontenc}  (loads iff Times font is used)\\
&\ey& \spfile{mathptmx} (loads iff exists \& Times font is used) \\
&\ey& \spfile{stmaryrd} (loads iff exists \& Times font is used)\\
&\ey& \spfile{textcomp} (loads iff Times font is used)\\
&\ey& \spfile{amssymb}  (loads iff Times font is used)\\
&\ey& \spfile{microtype} (loads iff present) \\
&\ey& \spfile{natbib} (loads unless experimental biblatex is used, cf. below)\\
&\ey& \spfile{inputenc} \\
&\ey& \spfile{xspace} \\
&\ey& \spfile{ifthen} \\
&\ey& \spfile{color} \\
&\ey& \spfile{hyperref} \\
&\ey& \spfile{amsmath} \\
&\ey& \spfile{ifpdf} \\
&\ey& \spfile{breakurl} (loads iff you use the \spcode{dvips} option) \\
&\ey& \spfile{graphicx}\\
&\ey& \spfile{subfigure}\\
&\ey& \spfile{float}}
%
If your document already calls any of these packages, then it is a
good idea to remove those calls from your preamble and let
\spfile{sp.cls} call them as it wants, since, sometimes, order
matters.

%=====================================================================
%=====================================================================

\section{Using the package}

%=====================================================================

\subsection{Loading the class file}

The package is loaded with
%
\begin{Verbatim}
\documentclass{sp}
\end{Verbatim}
%
The package insists on 12pt font. You should include the option
\spcode{dvips} if you are using postscript code (see section
\ref{sec:ps} for more on this, though).

The only other options recognized by the class concern fonts: the
default is Times (which can also be requested by the \spcode{times}
option). If you have problems with that, you can specifically resort
to the \LaTeX\ standard font, Computer Modern, with the \spcode{cm}
option. If you have the commercial Lucida fonts, which are the fonts
used by the journal for the final typesetting, you can request them
with the \spcode{lucida} option. The font options are, of course,
mutually exclusive.

%=====================================================================

\subsection{Loading the bst file}

References are handled with \BibTeX. \spfile{sp.bst} is designed with
the Web in mind, so, ideally, your bib database entries will include
doi and/or url information.
%
\begin{Verbatim}
\bibliography{your-bib-database}
\end{Verbatim}
%
Section \ref{sec:refs} covers references more fully. There we also
discuss the experimental biblatex support.

%=====================================================================

\subsection{Frontmatter}\label{sec:frontmatter}

\subsubsection{Metadata}\label{sec:metadata}

\SandP\ publications will appear on the Web in PDF format.  The
following metadata is useful for optimizing searches and the like:
%
\begin{Verbatim}
\pdfauthor{full author list}
\pdftitle{full title}
\pdfkeywords{keywords without special formatting, comma-delimited}
\end{Verbatim}
%
These commands should be given in the preamble.  Minimize the special
characters used there \dash this is basically ASCII-limited.


\subsubsection{Title and thanks}

The title is typeset in the usual way, but an initial argument inside
square brackets will insert a title in the running headers.  If your
title is longer than 30 characters, then this should be a short
version of your title.  Acknowledgments should be included with
\cmd{thanks} inside the \cmd{title} command.  Thus, the title command
looks like this:
%
\begin{Verbatim}
\title[Short title]%
  {Article title%
    \thanks{Many thanks to ...}%
  }
\end{Verbatim}


\subsubsection{Authors}

The \cmd{author} command can be used in the normal way.  But when 
typesetting the article for publication, the following format will
be used (so it is good to use it yourself as well):
%
\begin{Verbatim}
\author[Short authors]{%
  \spauthor{Author1 \\ \institute{Author1's Institute}} \AND
  \spauthor{Author2 \\ \institute{Author2's Institute}} \AND
  \spauthor{Author3 \\ \institute{Author3's Institute}}%
}
\end{Verbatim}
%
There is no limit to the number of authors you can have.  Each use of
\cmd{spauthor} should be separated by \cmd{AND}.  

The text inside square brackets appears in the running header. Use
full names for up to three authors and ``FirstAuthorLastName et al.''
for more than three.


\subsubsection{Abstract}

The \cmd{abstract} environment works in the same was as the standard
abstract environment for the \spcode{article} class:
%
\begin{Verbatim}
\begin{abstract}
  ...  
\end{abstract}
\end{Verbatim}


\subsubsection{Keywords}

The \cmd{keywords} environment is a standard one:
%
\begin{Verbatim}
\begin{keywords}
  ...  
\end{keywords}
\end{Verbatim}
%
Six is usually a good number for keywords, but the journal doesn't
impose restrictions here. You should use the same keywords both here
and in the pdfkeywords command discussed in Section~\ref{sec:metadata}.

%=====================================================================

\subsection{Article body}

We intend not to be overly controlling about how the body of the
article looks.  It's your work, after all, and we want to avoid taking
control of it.  But there are a few style guidelines that we would
like you to follow.


\subsubsection{Example sentences}\label{sec:examples}

You are free to use any package you wish for numbering examples.
However, we impose a few general formatting restrictions:
%
\example{
  \subexamples{
    \subexample{Example numbers appear in round parentheses, flush
      left.}
    \subexample{Subexamples are labeled with lowercase alphabetical
      letters, each followed by a period.}
    \subexample{Displayed equations (if any) and examples are numbered
      in the same sequence and spaced alike.}
    \subexample{References to examples appear inside parentheses, with
      no punctuation between elements.  So, for example, we can refer
      to~(\ref{ex:contents}), and also to its
      subexample~(\ref{ex:cls}).}
  }
}
\spfile{example.sty} takes care of this.  For example, here is a
displayed equation that you can compare the examples in this document
against:
%
\begin{equation}
  e^{\pi i} + 1 = 0.
\end{equation}
%
Furthermore, both examples and sub(sub)examples support \cmd{label}
and \cmd{ref}.

The source code for this document provides examples of
\spfile{example.sty} in action.  Here is the general format for a
complex case of nesting:
%
\begin{Verbatim}
\example{Optional general text for this block\\
  \subexamples{
    \subexample{An optional label for this subexample \\
      \subsubexamples{
        \subsubexample{Subsubexample}
        \subsubexample{Another one!}
      }
    }      
    \subexample{
      \subsubexamples{ 
        \subsubexample{\<{??}Questionable}
        \subsubexample{\<*Ungrammatical}
      }
    } 
  }
}
\end{Verbatim}
%
Group multiple examples together in \verb+\examples{...}+.  If you do
use \cmd{label} in an example or sub(sub)example, take care to avoid
surrounding stray spaces:
%
\begin{Verbatim}
\examples{
  \example{\label{ex:correct-1}This example will be
    typeset correctly.}
  \example{%
    \label{ex:correct-2}%
    This example will also be typeset correctly.}
  \example{
    \label{ex:incorrect}
    This example will have too much space.}
}
\end{Verbatim}
%
Page breaks are allowed between examples but not sub(sub)examples.

Because it is based on \spfile{tabularx.sty}, \spfile{example.sty}
lets you specify your own horizontal alignment in advanced cases.  It
also performs numbering independent of layout, so, for instance, you
can put multiple subexamples on a single line, or anywhere else.  For
example:

\examples[l@{}l@{}XMc]{
  \ex \label{ex:pseudoclefts}
      & \ey \label{ex:specificational}
      & The woman who saw her father is Mary.
      & \iota x.\, w(x) \land s(f(x),x) = M \\
      & \ey \label{ex:predicational}
      & The woman who saw her father is married to a politician.
      & m\bigl(\iota x.\, w(x) \land s(f(x),x)\bigr) \\
  \noalign{\addvspace\jot}
  \ex & \multicolumn{3}{@{}l@{}}
        {\ey \label{ex:ok}the woman who saw her father\quad
         \ey \label{ex:wco}\<*the woman who her father saw} \\
  \noalign{\addvspace\jot}
  (123) & \multicolumn{3}{@{}l@{}}
        {Mary wants to marry a custom example number.}}
Two kinds of pseudoclefts appear in~\eqref{ex:pseudoclefts}:
\eqref{ex:specificational} is specificational whereas
\eqref{ex:predicational} is predicational.
Both show the weak-crossover contrast between
\eqref{ex:ok} and~\eqref{ex:wco}.

\bigskip\noindent These examples are typeset as follows:
%
\begin{Verbatim}
\examples[l@{}l@{}XMc]{
 \ex \label{ex:pseudoclefts}
 & \ey \label{ex:specificational}
 & The woman who saw her father is Mary.
 & \iota x.\, w(x) \land s(f(x),x) = M \\
 & \ey \label{ex:predicational}
 & The woman who saw her father is married
   to a politician.
 & m\bigl(\iota x.\, w(x) \land s(f(x),x)\bigr) \\
 \noalign{\addvspace\jot}
 \ex & \multicolumn{3}{@{}l}
 {\ey \label{ex:ok}the woman who saw her father\quad
  \ey \label{ex:wco}\<*the woman who her father saw} \\
 \noalign{\addvspace\jot}
 (123) & \multicolumn{3}{@{}l@{}}
 {Mary wants to marry a custom example number.}}
\end{Verbatim}
%
The commands \cmd{ex}, \cmd{ey}, and \cmd{ez} produce consecutive
numbering for examples, subexamples, and subsubexamples, respectively.
They can be followed by \cmd{label}, as above.

By default, page breaks are permitted only where you use \cmd{noalign}
to add vertical space between rows, as above.  You can allow page
breaks between any two displayed rows by saying
\verb+\interdisplaylinepenalty=6999+ at the beginning of your paper.


\subsubsection{Citations}

The class file loads \spfile{natbib} for handling references, and it
also provides some additional macros for making citations easier.

Our guidelines for in-text references follow those of \emph{Linguistic
  Inquiry}:
%
\examples{
  \example{Page numbers are separated from the year by a nonbreaking
    space: \pgcitet{Montague74}{12}.}
  \example{Section and chapter numbers are separated from the year by a
    nonbreaking space and coded with $\S$:
    \seccitet{Montague74}{2}.}
  \example{References to articles are given without parentheses:
    \citealt{Montague74},
    \pgcitealt{Montague74}{12}, and so forth.}
  \example{References to an individual-as-author-of-a-text are given with
    the name followed by the year and other material inside
    parentheses: \citet{Montague74},
    \pgcitet{Montague74}{12}, and so forth.}
  \example{Possessive marking is on the name only:
    \posscitet{Montague74},
    \pgposscitet{Montague74}{12}, and so forth.}
  \example{Parenthetical references to articles do not contain
    parentheses of their own: \citep{Montague74},
    \pgcitep{Montague74}{12}, and so forth.}}
%
The following commands make this easy (the first four are standard
\spfile{natbib} commands):
%
\example{\subexamples{
  \subexample{\argcmd{citeauthor}{Montague74}  $\Rightarrow$ \citeauthor{Montague74}}
  \subexample{\argcmd{citealt}{Montague74}  $\Rightarrow$ \citealt{Montague74}}
  \subexample{\argcmd{citet}{Montague74}  $\Rightarrow$  \citet{Montague74}}
  \subexample{\argcmd{citep}{Montague74}  $\Rightarrow$  \citep{Montague74}}

  \subexample{\argcmd{posscitet}{Montague74}  $\Rightarrow$  \posscitet{Montague74}}
  \subexample{\argcmd{possciteauthor}{Montague74}  $\Rightarrow$  \possciteauthor{Montague74}}
  \subexample{\twoargcmd{pgposscitet}{Montague74}{988}  $\Rightarrow$  \pgposscitet{Montague74}{988}}
  \subexample{\twoargcmd{secposscitet}{Montague74}{2}  $\Rightarrow$  \secposscitet{Montague74}{2}}

  \subexample{\twoargcmd{pgcitealt}{Montague74}{988}  $\Rightarrow$ \pgcitealt{Montague74}{2}}
  \subexample{\twoargcmd{seccitealt}{Montague74}{2}  $\Rightarrow$  \seccitealt{Montague74}{2}}

  \subexample{\twoargcmd{pgcitep}{Montague74}{988}  $\Rightarrow$  \pgcitep{Montague74}{988}}
  \subexample{\twoargcmd{seccitep}{Montague74}{2}  $\Rightarrow$  \seccitep{Montague74}{2}}

  \subexample{\twoargcmd{pgcitet}{Montague74}{988}  $\Rightarrow$  \pgcitet{Montague74}{988}}
  \subexample{\twoargcmd{seccitet}{Montague74}{2}  $\Rightarrow$  \seccitet{Montague74}{2}}}}
%
We recommend that \emph{every} reference to an article be coded with
one of the variants of the \cmd{cite} command, to prevent spelling
mistakes, to accommodate global changes in formatting, and to avoid
in-text references without entries in the bibliography.


\subsubsection{Images}

The class file loads \spfile{graphicx}, so external files are easily
included with with commands like the following:
%
\begin{Verbatim}
\includegraphics[width=1in]{image-file.pdf}
\includegraphics[height=6cm]{image-file.jpg}
\end{Verbatim}
%
Users of \spfile{pdflatex} should use PDF or JPG images.  Users of
\spfile{dvips} should use EPS or PS.


\subsubsection{Floats (figures and tables)}

The class file formats figures and tables in a special way, using
macros built on the \spfile{float} package. To use this format, simply
format figures and tables in the usual way and let \spfile{sp.cls} do
its thing.

If you wish to deviate from this style, use the \spfile{float}
packages command \argcmd{floatstyle}{\ldots}, where \ldots\ is one of
the \spfile{float} package's accepted styles, along with one or both
of \argcmd{restylefloat}{table} and \argcmd{restylefloat}{figure}.

If you would like to restyle a single float (or group of them), use
brackets.  For example:

\begin{Verbatim}
{
  \floatstyle{plain}
  \restylefloat{figure}
  ...
  \begin{figure}
    ...
  \end{figure}
  ...
}
\end{Verbatim}


\subsubsection{Convenience macros}

The class file provides some macros that may be of use to our authors.

The colon in logical and set theoretic notation should not be typeset
with \spcode{``:''}, since that results in incorrect spacing
(\spcode{``:''} is interpreted as a relation symbol by \LaTeX) as in:
%
\example{$\forall x: x \in D \dots$}
%
Instead, one should use \verb?\colon\thinspace?, which results in much
better output:
%
\example{$\forall x\co x \in D \dots$}
%
The package provides the abbreviation \verb?\co?, which can be used
instead of \verb?\colon\thinspace?. So, to typeset the example above,
you would write \verb?$\forall x\co x \in D \dots$?.

For parenthetical remarks, the journal will use an em-dash --- like so
---, but we prefer if you insert the dash not with \verb?---? but with
the macro \verb?\dash?, which will result in better spacing \dash like
so \dash around the parenthesis.

If you would like to link to a webpage, you can use the \verb?\http?
macro, which will convert a string into a clickable link in the pdf
file. So, to link to \http{semprag.org}, you would write
\verb?\http{semprag.org}?. Similarly, mark up email addresses like
\email{editors@semprag.org} as \verb?\email{editors@semprag.org}?.

Finally, semanticists will appreciate that we provide a command
\verb?\sv? which produces semantic evaluation brackets:
\sv{\text{unicorn}}. Note that the command does not need to be used in
a math-environment, that is: it does not need to be surrounded by
\verb?$ ... $?, but it does create a math-environment for its
argument. So, if you want to use object language expressions inside
the evaluation brackets, you need to use \verb?\mbox? or \verb?\text?
(the latter is preferred). Thus, the above example was created by
using \verb?\sv{\text{unicorn}}?.

%=====================================================================

\subsection{Backmatter}

\subsubsection{Appendices}

Appendices are just sections inside the environment
%
\begin{Verbatim}
\begin{appendix}
  ...
\end{appendix}
\end{Verbatim}
%
This is a good home for lengthy proofs, fragments, experimental 
materials, and the like.


\subsubsection{References}\label{sec:refs}

References come after any appendices and just before the author
addresses section:
%
\begin{Verbatim}
\bibliography{your-bib-database}
\end{Verbatim}
%
where \spcode{your-bib-database} is the name of your bibtex database
file. For final typesetting, we will ask you to provide a
\spfile{bib}-file containing the bibtex entries for all your
references.

The class file uses the \BibTeX\ style file \spfile{sp.bst} as a
default.  This can be overridden with \cmd{bibliographystyle}, but
we'll use \spfile{sp.bst} for published versions. This new \BibTeX\
style file implements the guidelines of the ``Unified Stylesheet for
Linguistics'', which grew out of discussions among a group of editors
of linguistics journals during 2005-2006 and were approved on January
7, 2007. They were intended as a ``default, but with discretion to use
common sense", to quote David Denison.

To ensure that we can provide readers with the full bibliographic
information they deserve, please make sure that your bibtex database
satisfies the following:

\begin{enumerate}
\item Journal and book titles must be given in \emph{full} with initial
    letter of each major word capitalized.
\item Page references \emph{must} be given in \emph{full} for all
  articles in books and journals.
\item Use full first names of authors or editors.
\item In case of multiple authorship, the names of all authors must be
  given.
\item When possible provide the issue number and not just the volume
  number for a journal article.
\item Provide the doi number of a journal article whenever
  possible. If the information is not directly available with the
  article, use the form at
  \http{crossref.org/SimpleTextQuery/} to find the doi.
\item For conference proceedings title, use the name of the society
      and then put the meeting's acronym in parentheses. Otherwise treat
      as a journal article. Do not include the words ``proceedings of
      the'' or ``papers from the''.  
      
\end{enumerate}

\paragraph{Experimental biblatex bibliography style:}
%
The \texttt{sp-latex} package now includes a biblatex bibliography style 
more fully implementing the unified stylesheet: 
\texttt{sp-biblatex.bbx}. If you would like to use this, add the class 
option \texttt{biblatex} when you call \texttt{sp.cls}:
%
\begin{Verbatim}
\documentclass[biblatex]{sp}
\end{Verbatim}
%
Also, add a line to the preamble that loads your \spfile{bib}-file: 
%
\begin{Verbatim}
\addbibresource{your-bib-database.bib}
\end{Verbatim}
%
Finally, replace the \cmd{bibliography} line in the backmatter with the 
following:
%
\begin{Verbatim}
\printbibliography
\end{Verbatim}
%
Otherwise the transition should be seamless. Eventually, we will move to 
biblatex completely. If you encounter issues, please email us: 
\email{latex@semprag.org}.

\subsubsection{Author addresses}

Full author addresses appear at the end of each article.  They are
specified as follows:
%
\begin{Verbatim}
\begin{addresses}
  \begin{address}
    Author1 \\
    Street \\
    ... \\
    \email{author1@email}
  \end{address}
  \begin{address}
    Author2 \\
    Street \\
    ... \\
    \email{author2@email}
  \end{address}
  ...
\end{addresses}
\end{Verbatim}


%=====================================================================

\subsection{A note on using postscript}\label{sec:ps}

We discourage using postscript, and we in turn strongly encourage
using \spfile{pdflatex} or one of its sibling to generate your output.
As far as we know, only the direct route to PDF can ensure that line
breaks happen where they are supposed to and hyperlinks work properly.

We understand that you might depend upon postscript for your diagrams.
If this is so, then we suggest the following:
%
\begin{enumerate}
\item Create PDF files of your diagrams and insert them with a command
  like \cmd{includegraphics}. (The class file loads \spfile{graphicx}
  already.)
\item Use \spfile{pdftricks} or one of its variants.  (This might be
  somewhat arduous though.)
\end{enumerate}
%
If neither of these options works for you, please contact us.  We are
happy to work with you to ensure that your diagrams look the way you
want them to.

With all that said, it is worth pointing out that including the option
\spcode{dvips} in your \spcode{documentclass} specification will
instruct \spfile{hyperref} to deal with your hyperlinks and the like
as best it can.
%
\begin{Verbatim}
\documentclass[dvips]{sp}
\end{Verbatim}
%
This will also call \spfile{breakurl} into action to assist.  This
might produce first-rate output if your document contains few
hyperlinks.

%=====================================================================

\section{Troubleshooting}

If you have any problems with the journal's \LaTeX\ package, please
send email to \email{latex@semprag.org}.

In our experience, there may be font selection issues, in which case
we recommend trying the \spcode{cm} class option, which uses the 
standard Computer Modern fonts rather than Times.

%=====================================================================

\bibliography{guide}

%=====================================================================

\begin{addresses}
  \begin{address}
    Kai von Fintel\\
    Department of Linguistics \& Philosophy\\
    Massachusetts Institute of Technology\\
    77 Massachusetts Avenue, 32-D808\\
    Cambridge, MA 02139\\
    USA\\
    \email{fintel@mit.edu}
  \end{address}
  \begin{address}
    Christopher Potts\\
    Department of Linguistics\\
    226 South College\\
    150 Hicks Way\\
    Amherst, MA 01003\\
    USA\\
    \email{potts@linguist.umass.edu}
  \end{address}
  \begin{address}
    Chung-chieh Shan\\
    Department of Computer Science and Center for Cognitive Science\\
    Rutgers University\\
    110 Frelinghuysen Rd\\
    Piscataway, NJ 08854-8019\\
    USA\\
    \email{ccshan@rutgers.edu}
  \end{address}
\end{addresses}

%=====================================================================

\end{document}

